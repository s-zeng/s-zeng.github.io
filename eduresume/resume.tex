%%%%%%%%%%%%%%%%%
% This is an sample CV template created using altacv.cls
% (v1.1.4, 27 July 2018) written by LianTze Lim (liantze@gmail.com). Now compiles with pdfLaTeX, XeLaTeX and LuaLaTeX.
%
%% It may be distributed and/or modified under the
%% conditions of the LaTeX Project Public License, either version 1.3
%% of this license or (at your option) any later version.
%% The latest version of this license is in
%%    http://www.latex-project.org/lppl.txt
%% and version 1.3 or later is part of all distributions of LaTeX
%% version 2003/12/01 or later.
%%%%%%%%%%%%%%%%

%% If you need to pass whatever options to xcolor
\PassOptionsToPackage{dvipsnames}{xcolor}

%% If you are using \orcid or academicons
%% icons, make sure you have the academicons
%% option here, and compile with XeLaTeX
%% or LuaLaTeX.
% \documentclass[10pt,a4paper,academicons]{altacv}

%% Use the "normalphoto" option if you want a normal photo instead of cropped to a circle
% \documentclass[10pt,a4paper,normalphoto]{altacv}

\documentclass[10pt,a4paper]{altacv}
% \documentclass[10pt,letter]{altacv}

%% AltaCV uses the fontawesome and academicon fonts
%% and packages.
%% See texdoc.net/pkg/fontawecome and http://texdoc.net/pkg/academicons for full list of symbols.
%%
%% Compile with LuaLaTeX for best results. If you
%% want to use XeLaTeX, you may need to install
%% Academicons.ttf in your operating system's font
%% folder.


% Change the page layout if you need to
\geometry{left=0.75cm,right=7.55cm,marginparwidth=6.35cm,marginparsep=0.55cm,top=1cm,bottom=1cm}

\newenvironment{tightitemize} % Defines the tightitemize environment which modifies the itemize environment to be more compact
{\begin{itemize}\itemsep1pt \parskip0pt \parsep0pt}
{\end{itemize}\vspace{-\topsep}}
% Change the font if you want to.

% If using pdflatex:
\usepackage[utf8]{inputenc}
\usepackage{amssymb}
\usepackage[T1]{fontenc}
\usepackage{hyperref}
\usepackage{fancybox}
% \usepackage[default]{lato}

% If using xelatex or lualatex:
\setmainfont{Lato}

% Change the colours if you want to
\definecolor{Mulberry}{HTML}{92243D}
\definecolor{Cool}{HTML}{E2243D}
\definecolor{SlateGrey}{HTML}{2E2E2E}
\definecolor{LightGrey}{HTML}{666666}
\colorlet{heading}{Cool}
\colorlet{accent}{Mulberry}
\colorlet{emphasis}{SlateGrey}
\colorlet{body}{LightGrey}

% Change the bullets for itemize and rating marker
% for \cvskill if you want to
\renewcommand{\itemmarker}{{\small\textbullet}}
\renewcommand{\ratingmarker}{\faCircle}

\hypersetup{
    pdfauthor={Simon Zeng},
    pdftitle={Simon Zeng - Resume},
    colorlinks = false,
    allbordercolors=white
}

%% sample.bib contains your publications
\addbibresource{sample.bib}

\begin{document}
\name{Simon Zeng}
\tagline{Software Engineer}
% \photo{2.8cm}{pic2}
\personalinfo{%
  % Not all of these are required!
  % You can add your own with \printinfo{symbol}{detail}
  \email{\href{mailto:contact@simonzeng.com}{contact@simonzeng.com}}
  \phone{1 (613) 983-9079}
  % \mailaddress{58 Akenhead Crescent, Kanata, ON, K2T0B4}
  \homepage{\href{http://simonzeng.com}{simonzeng.com}}
  % \twitter{@twitterhandle}
  \linkedin{\href{https://linkedin.com/in/s-zeng1}{s-zeng1}}
  \github{\href{https://github.com/s-zeng}{s-zeng}}
  % \printinfo{UW}{20769883}
  %% You MUST add the academicons option to \documentclass, then compile with LuaLaTeX or XeLaTeX, if you want to use \orcid or other academicons commands.
%   \orcid{orcid.org/0000-0000-0000-0000}
}

%% Make the header extend all the way to the right, if you want.
\begin{fullwidth}
\makecvheader
\end{fullwidth}

%% Depending on your tastes, you may want to make fonts of itemize environments slightly smaller
% \AtBeginEnvironment{itemize}{\small}

%% Provide the file name containing the sidebar contents as an optional parameter to \cvsection.
%% You can always just use \marginpar{...} if you do
%% not need to align the top of the contents to any
%% \cvsection title in the "main" bar.
\cvsection[sidebar]{Experience}
% \cvsection{Experience}

\job{TQ Tezos}{Software Engineering Intern 
(Blockchain)}{\skills{Blockchain}{OCaml}{}}{Sep 2020 -- Dec 2020}{New York, New York}
\begin{tightitemize}
    \item Contributed to Python, Haskell, and OCaml blockchain products
    \item Authored and formally verified Tezos smart contracts in an OCaml-based 
        DSL to automate proprietary applications for private blockchains
    \item Developed and instrumented automated deployment of test and private 
        chains on Linux machines over Kubernetes
    \item Constructed peer-to-peer staking topology monitoring and visualization 
        infrastructure with Python
\end{tightitemize}

\smallskip
\divider
% \smallskip


\job{Tesla}{Software Engineering Intern (Firmware 
Tooling)}{\skills{Haskell}{Python}{Java}}{Jan 2020 -- Aug 2020}{Palo Alto, California}
\begin{tightitemize}
\item Developed and maintained large Haskell code base responsible for automated firmware 
    documentation, code, and signal generation
\item Improved Haskell products' performance and runtimes by over 20\%
\item Responsible for design and implementation of core firmware
    verification infrastructure employed by entire organization
\item Designed and developed robust firmware signal inspection architecture with 
    Java and modern statically typed Python
\end{tightitemize}

\smallskip
\divider
% \smallskip

\job{University of Waterloo}{Teaching Assistant (Algebra)}{\skills{Pure 
Math}{Teaching}{}}{Sep 2019 -- Dec 2019}{Waterloo, Ontario}
\begin{tightitemize}
\item Tutored classes of over 1000 students in number theory and abstract algebra
\item Prepared individual tutoring lesson plans to ameliorate 
    understanding in advanced topics such as quadratic reciprocity or 
    interactive theorem proving
\end{tightitemize}

\smallskip
\divider
% \smallskip

\job{Ericsson}{Software Engineering Intern 
(Performance)}{\skills{Clojure}{Python}{}}{May 2019 -- Aug 2019}{Kanata, Ontario}
\begin{tightitemize}
    \item Developed pure functional Clojure metrics infrastructure to monitor 
        complex JVM architectures, allowing for discovery of multiple 
        performance issues
    \item Implemented a parser and interpreter for an internally designed domain 
        specific language specifically tailored for highly variable telecom 
        network performance evaluation environments, greatly reducing workload 
        for performance team
    \item Contributed to Python performance testing applications
\end{tightitemize}

\smallskip
\divider
% \smallskip

\job{CENX}{Software Engineering Intern (Test 
Automation)}{\skills{Python}{Security}{}}{Jul 2017 -- Sep 2017}{Ottawa, Ontario}
\begin{tightitemize}
    \item Developed robust automated Python framework for load-testing web apps
    \item Created custom implementation of IETF RFC socket protocols to debug 
        non-standard network stacks
    \item Discovered multiple security issues, including cryptography 
        weaknesses, via automated fuzzing
\end{tightitemize}

\smallskip
\divider
% \smallskip

\job{inBay Technologies}{Software Engineering Intern (Full 
Stack)}{\skills{Ruby}{Rails}{Javascript}}{Jul 2016 -- Aug 2016}{Kanata, Ontario}
\begin{tightitemize}
\item Created internal use development tools backed by Ruby on Rails 
    and Javascript to monitor and debug specialized production systems
\end{tightitemize}

% \cvsection{A Day of My Life}

% Adapted from @Jake's answer from http://tex.stackexchange.com/a/82729/226
% \wheelchart{outer radius}{inner radius}{
% comma-separated list of value/text width/color/detail}
% \wheelchart{1.5cm}{0.5cm}{%
  % 6/8em/accent!30/{Sleep,\\beautiful sleep},
  % 3/8em/accent!40/Hopeful novelist by night,
  % 8/8em/accent!60/Daytime job,
  % 2/10em/accent/Sports and relaxation,
  % 5/6em/accent!20/Spending time with family
% }

% \clearpage
% \cvsection[page2sidebar]{Publications}

% \nocite{*}

% \printbibliography[heading=pubtype,title={\printinfo{\faBook}{Books}},type=book]

% \divider

% \printbibliography[heading=pubtype,title={\printinfo{\faFileTextO}{Journal Articles}},type=article]

% \divider

% \printbibliography[heading=pubtype,title={\printinfo{\faGroup}{Conference Proceedings}},type=inproceedings]

%% If the NEXT page doesn't start with a \cvsection but you'd
%% still like to add a sidebar, then use this command on THIS
%% page to add it. The optional argument lets you pull up the
%% sidebar a bit so that it looks aligned with the top of the
%% main column.
% \addnextpagesidebar[-1ex]{page3sidebar}


\end{document}
