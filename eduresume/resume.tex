%%%%%%%%%%%%%%%%%
% This is an sample CV template created using altacv.cls
% (v1.1.4, 27 July 2018) written by LianTze Lim (liantze@gmail.com). Now compiles with pdfLaTeX, XeLaTeX and LuaLaTeX.
%
%% It may be distributed and/or modified under the
%% conditions of the LaTeX Project Public License, either version 1.3
%% of this license or (at your option) any later version.
%% The latest version of this license is in
%%    http://www.latex-project.org/lppl.txt
%% and version 1.3 or later is part of all distributions of LaTeX
%% version 2003/12/01 or later.
%%%%%%%%%%%%%%%%

%% If you need to pass whatever options to xcolor
\PassOptionsToPackage{dvipsnames}{xcolor}

%% If you are using \orcid or academicons
%% icons, make sure you have the academicons
%% option here, and compile with XeLaTeX
%% or LuaLaTeX.
% \documentclass[10pt,a4paper,academicons]{altacv}

%% Use the "normalphoto" option if you want a normal photo instead of cropped to a circle
% \documentclass[10pt,a4paper,normalphoto]{altacv}

\documentclass[10pt,a4paper]{altacv}

%% AltaCV uses the fontawesome and academicon fonts
%% and packages.
%% See texdoc.net/pkg/fontawecome and http://texdoc.net/pkg/academicons for full list of symbols.
%%
%% Compile with LuaLaTeX for best results. If you
%% want to use XeLaTeX, you may need to install
%% Academicons.ttf in your operating system's font
%% folder.


% Change the page layout if you need to
\geometry{left=1cm,right=8cm,marginparwidth=6.0cm,marginparsep=0.8cm,top=1.20cm,bottom=1.00cm}

\newenvironment{tightitemize} % Defines the tightitemize environment which modifies the itemize environment to be more compact
{\vspace{-\topsep}\begin{itemize}\itemsep1pt \parskip0pt \parsep0pt}
{\end{itemize}\vspace{-\topsep}}
% Change the font if you want to.

% If using pdflatex:
\usepackage[utf8]{inputenc}
\usepackage{amssymb}
\usepackage[T1]{fontenc}
\usepackage{hyperref}
\usepackage{fancybox}
% \usepackage[default]{lato}

% If using xelatex or lualatex:
\setmainfont{Lato}

% Change the colours if you want to
\definecolor{Mulberry}{HTML}{92243D}
\definecolor{Cool}{HTML}{E2243D}
\definecolor{SlateGrey}{HTML}{2E2E2E}
\definecolor{LightGrey}{HTML}{666666}
\colorlet{heading}{Cool}
\colorlet{accent}{Mulberry}
\colorlet{emphasis}{SlateGrey}
\colorlet{body}{LightGrey}

% Change the bullets for itemize and rating marker
% for \cvskill if you want to
\renewcommand{\itemmarker}{{\small\textbullet}}
\renewcommand{\ratingmarker}{\faCircle}

\hypersetup{
    pdfauthor={Simon Zeng},
    pdftitle={Simon Zeng - Resume},
    colorlinks = false,
    allbordercolors=white
}

%% sample.bib contains your publications
\addbibresource{sample.bib}

\begin{document}
\name{Simon Zeng}
\tagline{Software Engineer}
% \photo{2.8cm}{pic2}
\personalinfo{%
  % Not all of these are required!
  % You can add your own with \printinfo{symbol}{detail}
  \email{\href{mailto:contact@simonzeng.com}{contact@simonzeng.com}}
  \phone{US: 408-429-9773}
  % \mailaddress{58 Akenhead Crescent, Kanata, ON, K2T0B4}
  \homepage{\href{http://simonzeng.com}{simonzeng.com}}
  % \twitter{@twitterhandle}
  \linkedin{\href{https://linkedin.com/in/s-zeng1}{s-zeng1}}
  \github{\href{https://github.com/s-zeng}{s-zeng}}
  % \printinfo{UW}{20769883}
  %% You MUST add the academicons option to \documentclass, then compile with LuaLaTeX or XeLaTeX, if you want to use \orcid or other academicons commands.
%   \orcid{orcid.org/0000-0000-0000-0000}
}

%% Make the header extend all the way to the right, if you want.
\begin{fullwidth}
\makecvheader
\end{fullwidth}

%% Depending on your tastes, you may want to make fonts of itemize environments slightly smaller
% \AtBeginEnvironment{itemize}{\small}

%% Provide the file name containing the sidebar contents as an optional parameter to \cvsection.
%% You can always just use \marginpar{...} if you do
%% not need to align the top of the contents to any
%% \cvsection title in the "main" bar.
\cvsection[sidebar]{Experience}
% \cvsection{Experience}

\cvevent{\textcolor{Mulberry}{\textbf{Tesla Inc}} \textcolor{LightGrey}{{::}} 
Software Engineering Intern (Firmware Tooling)
}{\textrm{\textcolor{Mulberry!75}{\normalfont \cvtag{Haskell} \cvtag{Python} 
\cvtag{Java}}}}{January 2020 -- 
April 
2020}{Palo Alto, California}
\vspace{\topsep} % Hacky fix for awkward extra vertical space
\begin{tightitemize}
\item Developed and maintained large Haskell code base responsible for automated firmware 
    documentation and code generation
\item Instrumented and completed an extensive porting undertaking from legacy 
    Haskell codebases to modern statically typed Python
\item Responsible for design and implementation of core version control 
    verification infrastructure employed by entire firmware organization
\end{tightitemize}

\smallskip
\divider
\smallskip

\cvevent{\textcolor{Mulberry}{\textbf{Ericsson}} \textcolor{LightGrey}{{::}}
Software Engineering Intern (Performance)}{\textrm{\textcolor{Mulberry!75}{\normalfont \cvtag{Clojure} 
\cvtag{Python}}}}{May 2019 -- August 2019}{Kanata, Ontario}
\vspace{\topsep} % Hacky fix for awkward extra vertical space
\begin{tightitemize}
    \item Developed pure functional Clojure metrics infrastructure to monitor 
        complex JVM architectures
    \item Implemented a parser and interpreter in Python for an 
        internally designed domain specific language specifically tailored for 
        highly variable performance evaluation environments
\end{tightitemize}

\smallskip
\divider
\smallskip

\cvevent{\textcolor{Mulberry}{\textbf{CENX Inc}} \textcolor{LightGrey}{{::}}
Software Engineering Intern (Test Automation)}{\textrm{\textcolor{Mulberry!75}{\normalfont \cvtag{Python} 
\cvtag{Networking} \cvtag{Security}}}}{July 2017 -- September 2017}{Ottawa, Ontario}
% Developed low-level network interface technologies in order to maximize artificial 
% demand/stress on company products for the purpose of augmenting performance, 
% assuring quality, and detecting security issues.
\vspace{\topsep} % Hacky fix for awkward extra vertical space
\begin{tightitemize}
    \item Developed robust automated UI-testing Python framework for 
        load-testing web applications
    \item Created custom implementation of IETF RFC socket protocols to debug 
        non-standard network stacks
    \item Discovered multiple security issues, including cryptography 
        weaknesses, via automated fuzzing
\end{tightitemize}

\smallskip
\divider
\smallskip

\cvevent{\textcolor{Mulberry}{\textbf{inBay Technologies}} \textcolor{LightGrey}{{::}}
Software Engineering Intern (Full Stack)}{\textrm{\textcolor{Mulberry!75}{\normalfont 
\cvtag{Ruby} \cvtag{Rails} \cvtag{Javascript} \cvtag{HTML/CSS}}}}{July 2016 -- 
August 2016}{Kanata, Ontario}
% Created internal-use development tools, performing both front-end and back-end 
% development using web technologies such as Ruby on Rails and node.js
\vspace{\topsep} % Hacky fix for awkward extra vertical space
\begin{tightitemize}
\item Created internal use development tools backed by Ruby on Rails 
    and NodeJS to monitor and debug specialized production 
    systems
\end{tightitemize}

\smallskip
\divider
\smallskip

\cvevent{\textcolor{Mulberry}{\textbf{University of Waterloo}} \textcolor{LightGrey}{{::}}
Algebra Teaching Assistant}{\textrm{\textcolor{Mulberry!75}{\normalfont 
\cvtag{Number Theory} \cvtag{Abstract Algebra} \cvtag{Formal Logic}}}}{September 2019 -- 
December 2019}{Waterloo, Ontario}
% Created internal-use development tools, performing both front-end and back-end 
% development using web technologies such as Ruby on Rails and node.js
\vspace{\topsep} % Hacky fix for awkward extra vertical space
\begin{tightitemize}
\item Tutored classes of over 1000 students in number theory, abstract algebra, 
    and formal logic
\item Prepared careful individual tutoring lesson plans to ameliorate 
    understanding in advanced topics such as quadratic reciprocity or 
    interactive theorem proving
\end{tightitemize}

% \cvevent{Volunteer STEM Camp Counsellor}{Virtual Ventures}{July 2015 -- August 2015}{Ottawa, Ontario}
% Taught lessons and lead activities focusing on programming, robotics, and 
% physics for children aged 9-12

% \medskip

% \cvsection{Projects}

% \cvevent{Board To 
% Latex}{\href{https://github.com/s-zeng/board-to-latex}{github.com/s-zeng/board-to-latex}}{}{}
% \vspace{\topsep} % Hacky fix for awkward extra vertical space
% \begin{tightitemize}
%     \item \textbf{React} webapp that transcribes photos of chalkboards and whiteboards to 
%         latex, with OCR handled by a \textbf{Flask} backend
% \end{tightitemize}

% \divider

% \cvevent{Machine Learning Ragtime 
% Generator}{\href{https://github.com/s-zeng/rag-shenanigann}{github.com/s-zeng/rag-shenanigann}}{}{}
% \vspace{\topsep} % Hacky fix for awkward extra vertical space
% \begin{tightitemize}
%     \item A suite of \textbf{Python} scripts that scrape ragtime MIDI files from 
%         the web, and preprocesses them into a custom machine-readable format 
%         designed for easy neural network training, then converts model output 
%         back to midi
% \end{tightitemize}

% \divider

% \cvevent{Tiny Polynomial Interpolator}{\href{https://github.com/s-zeng/interpoly}{github.com/s-zeng/interpoly}}{}{}
% \vspace{\topsep} % Hacky fix for awkward extra vertical space
% \begin{tightitemize}
%     \item Extremely small polynomial interpolating CLI tool written in 15 lines (447 
%         bytes) of \textbf{Haskell}
%     \item Uses a custom technique with better precision than Lagrange 
%         interpolation
% \end{tightitemize}

% \divider

% \cvevent{Discord API Music Player 
%     Bot}{\href{https://github.com/s-zeng/Zengyatta}{github.com/s-zeng/Zengyatta}}{}{}
% \begin{itemize}
%     \item Interfaced with the Discord REST API using \textbf{Java} to create a 
%         bot that plays music in voice channels
%     \item Reverse engineered Bandcamp's web services to enable the use of 
%         Bandcamp links as a music source
% \end{itemize}


% \cvevent{repl.vim}{github.com/ujihisa/repl.vim}{}{}
% Vim plugin that provides bindings for summoning an interactive environment with 
% the code that is being written. My contributions include additional code to make 
% the plugin compatible with Racket/Scheme source files and interactive environments

\smallskip

% \cvsection{A Day of My Life}

% Adapted from @Jake's answer from http://tex.stackexchange.com/a/82729/226
% \wheelchart{outer radius}{inner radius}{
% comma-separated list of value/text width/color/detail}
% \wheelchart{1.5cm}{0.5cm}{%
  % 6/8em/accent!30/{Sleep,\\beautiful sleep},
  % 3/8em/accent!40/Hopeful novelist by night,
  % 8/8em/accent!60/Daytime job,
  % 2/10em/accent/Sports and relaxation,
  % 5/6em/accent!20/Spending time with family
% }

% \clearpage
% \cvsection[page2sidebar]{Publications}

% \nocite{*}

% \printbibliography[heading=pubtype,title={\printinfo{\faBook}{Books}},type=book]

% \divider

% \printbibliography[heading=pubtype,title={\printinfo{\faFileTextO}{Journal Articles}},type=article]

% \divider

% \printbibliography[heading=pubtype,title={\printinfo{\faGroup}{Conference Proceedings}},type=inproceedings]

%% If the NEXT page doesn't start with a \cvsection but you'd
%% still like to add a sidebar, then use this command on THIS
%% page to add it. The optional argument lets you pull up the
%% sidebar a bit so that it looks aligned with the top of the
%% main column.
% \addnextpagesidebar[-1ex]{page3sidebar}


\end{document}
